%% The first command in your LaTeX source must be the \documentclass command.
%%
%% Options:
%% twocolumn : Two column layout.
%% hf: enable header and footer.
\documentclass[
% twocolumn,
% hf,
]{ceurart}

%%
%% One can fix some overfulls
\sloppy

%%
%% Minted listings support 
%% Need pygment <http://pygments.org/> <http://pypi.python.org/pypi/Pygments>
\usepackage{listings}
\usepackage{lmodern}
\usepackage{minted}
\usepackage{textcomp}
%% auto break lines
\lstset{breaklines=true}

%%
%% end of the preamble, start of the body of the document source.
\begin{document}

%%
%% Rights management information.
%% CC-BY is default license.
\copyrightyear{2022}
\copyrightclause{Copyright for this paper by its authors.
  Use permitted under Creative Commons License Attribution 4.0
  International (CC BY 4.0).}

%%
%% This command is for the conference information
\conference{IJCAI-ECAI 2022, the 31st International Joint Conference on Artificial Intelligence and the 25th European Conference on Artificial Intelligence, July 23-29, 2022
Messe Wien, Vienna, Austria}

%%
%% The "title" command
\title{Understand your clusters: a link between the data and metadata}

\tnotemark[1]
\tnotetext[1]{You can use this document as the template for preparing your
  publication. We recommend using the latest version of the ceurart style.}

%%
%% The "author" command and its associated commands are used to define
%% the authors and their affiliations.
\author[1,2]{Dmitry S. Kulyabov}[%
orcid=0000-0002-0877-7063,
email=kulyabov-ds@rudn.ru,
url=https://yamadharma.github.io/,
]
\cormark[1]
\fnmark[1]
\address[1]{Peoples' Friendship University of Russia (RUDN University),
  6 Miklukho-Maklaya St, Moscow, 117198, Russian Federation}
\address[2]{Joint Institute for Nuclear Research,
  6 Joliot-Curie, Dubna, Moscow region, 141980, Russian Federation}

\author[3]{Ilaria Tiddi}[%
orcid=0000-0001-7116-9338,
email=i.tiddi@vu.nl,
url=https://kmitd.github.io/ilaria/,
]
\fnmark[1]
\address[3]{Vrije Universiteit Amsterdam, De Boelelaan 1105, 1081 HV Amsterdam, The Netherlands}

%% Footnotes
\cortext[1]{Corresponding author.}
\fntext[1]{These authors contributed equally.}

%%
%% The abstract is a short summary of the work to be presented in the
%% article.
\begin{abstract}
  In this preliminary work we present an approach for clustering augmented with Natural Language Explanations.
  With clustering there are 2 main challenges: a) choice of proper, reasonable number of clusters and b) cluster profiling.
  There is a plethora of technics for a) but not so much for b), which is in general laborious task of explaining obtained clusters.
  Clustering is in a sense art in that regard that it is intuitive and iterative process.
  Therefore, XAI techniques are well suited in this area.
  On a convincing example we show how process of clustering on a set of "objective" variables could be facilitated with textual metadata.
  In our case images of products from online fashion store are used for clustering.
  Then product descriptions are used for profiling clusters.
\end{abstract}

%%
%% Keywords. The author(s) should pick words that accurately describe
%% the work being presented. Separate the keywords with commas.
\begin{keywords}
  XAI \sep
  NLU \sep
  clustering \sep
  TODO
\end{keywords}

%%
%% This command processes the author and affiliation and title
%% information and builds the first part of the formatted document.
\maketitle

\section{Introduction}
%In general what is the prroblem -- in clustering we need to analyse results, XAI can help us in doing that
- classification as a tool for understanding
- the problem is common in fields like: marketing, e-commerce, industries, ???
- one like to classify "objective" data like images, sensors, etc to uncover hidden structure
   - to many observations to do it manually
- the difficulty arises to explain what algorithm has learned and if it is of any value
- explanations needs to be done with data comprehensible by humans
  - images - OK, but not always (vide images of medical images, satellites,  etc)
  - industrial data - rarely
  - thus "metadata" - link between "objective/hard" data and "soft" - like labels given by humans eg descriptions of products
  - higher probability of bias
  - sanity check of metadata
- one want to have influence on final outcome
  - as it is exploratory, it is unsupervised
  - no direct influence -> to adress this ...

\section{Related works}
%How this is done in the field

\section{Interactive explanation creation}
%Description of the algorithms, results on synmthetic data
- Usually clustering consists of 2 steps:
  - choice of number of clusters
  - profiling
  - repetitive process
- we propose to explain clusters derived from 1 modality ("hard data") with textual modality (usual form of metadata)
  - for choice of clusters: TSNE and silhouette score
  - then textual explanations follows straight away
  - influence on explanations: 1) stopwords 2) whitelist


\section{Real case scenario}
%About the paper citation clustering  and what are plans for extending that part.
In this part we will show how our framework could be applied to real case scenario.
We choose example from e-commerce filed because authors have experience of working in this industry.



\section{Summary}
SEO
visual modality
- hierarchical clustering
- Latent Dirichlet Allocation
%:)

\begin{acknowledgments}
  Thanks to the developers of ACM consolidated LaTeX styles
  \url{https://github.com/borisveytsman/acmart} and to the developers
  of Elsevier updated \LaTeX{} templates
  \url{https://www.ctan.org/tex-archive/macros/latex/contrib/els-cas-templates}.  
\end{acknowledgments}

%%
%% Define the bibliography file to be used
\bibliography{sample-ceur}





\end{document}

%%
%% End of file
